\documentclass[conference]{acmsiggraph}

\usepackage{mdwlist}
\usepackage{graphics}
\usepackage{graphicx}
\usepackage{listings}
\usepackage{enumitem}
\usepackage{amsmath}
\usepackage[utf8]{inputenc}

\newcommand{\onecol}[3]{
  \begin{figure}
    \centering
    \includegraphics[width=\columnwidth]{#1}
    \vspace{0.0in}%??
    \caption{#3}
    \label{#2}
  \end{figure}
}

\newcommand{\twocol}[3]{
  \begin{figure*}
    \centering
    \includegraphics[width=5.5in]{#1}
    \vspace{0.1in}%??
    \caption{#3}
    \label{#2}
  \end{figure*}
}

%%%%%%%%%%%%%%%
% Information %
%%%%%%%%%%%%%%%

\title{Plant Modeling Using Environmental Parameters}

\author{
  Andre Philipp\thanks{e-mail:philipp.andy@gmail.com}
  \qquad
  Andy Spencer\thanks{e-mail:andy753421@ucla.edu}
  \qquad
  Haakon Garseg Mørk\thanks{e-mail:haakongmork@gmail.com}
  \\
  University of California, Los Angeles \\
  CS275 Artificial Life for Computer Graphics and Vision
}

\pdfauthor{Andre Philipp, Andy Spencer, Haakon Garseg Mørk}

\keywords{radiosity, global illumination, constant time}

%%%%%%%%%%%%
% Document %
%%%%%%%%%%%%

% Proposal
% --------
% Team:
%
%   Andy Philipp <philipp.andy@gmail.com>
%   Andy Spencer <andy753421@ucla.edu>
%   Haakon Garseg Moerk <haakongmork@gmail.com>
%
% Title:
%
%   Plant Modeling Using Environmental Parameters
%
% Summary:
%
%   The project will focus on modeling natural looking plants using
%   simulated environment parameters to control growth and evolution of
%   different plant species. Genetic algorithms or other evolutionary
%   techniques will be used to evolve plant suitable for growth in various
%   environments.
%
% Potential environmental conditions:
%
%   - Availability of water
%   - Amount of sunlight
%   - Soil conditions
%   - Growth elevation
%
% Potential plant species parameters:
%
%   - Rate of water loss
%   - Food and water storage capability
%   - Total mass/volume
%   - Crown height
%   - Leaf area
%   - Root structures

% Project submission
% ------------------
% We need to make web page containing:
%
%   1. The project abstract
%   2. A link to the paper
%   3. Links to any demos or videos
%
% The paper should be in conference format:
%
%   1. We can use two-column style with, e.g.:
%        a) references,  b) technical description,
%        c) results,     d) conclusions
%   2. It should be medium length, 5 pages is fine.

% Outline
% -------
% 1. Introduction
% 2. Architecture
% 3. Plant System
% 4. Genetics
% 5. Results
% 6. Conclusions

\begin{document}

%%%%%%%%%%%%%%
% Title Page %
%%%%%%%%%%%%%%

\maketitle

\begin{abstract}

L-systems, or Lindenmayer systems, are often used to create realistic looking
artificial plant environments. While conceptually simple, L-systems have been
extended with many features to create realistic plants models based on a
description of the plant growth patterns. Significant progress has been made in
providing robust frameworks for modeling artificial plants, but the process of
creating specific plants species is still very much a manual process that often
requires a deep understanding of the tools used.

In this project we will explore an alternate approach to plant species creation
using genetic algorithms to automatically evolve plants suitable to particular
environments.

By tuning environmental parameters for the availability of water, the amount of
sunlight, and the ambient soil conditions, artificial plant species are produced
which resembling physical plants that may grown in similar environments.

With this approach to plant species generation, the underlying L-system
production rules are evolved based on random mutation combined with a fitness
function to determine plant survival rates. In addition to the L-systems,
several additional parameters are introduced to control plant growth and
simulation at a more
abstract level.

\end{abstract}

\keywordlist

%%%%%%%%%%%%
% Content %
%%%%%%%%%%%%

\section{Software Architecture}

We closely examined several existing frameworks that we considered using for
this project. The two frameworks that were the best candidates were Algorithmic
Botany\cite{abotany} and OpenAlea\cite{openalea}.

The Algorithmic Botany suit consists of the VLAB software which can be used on
Mac OS and Linux, the L-studio software for Microsoft Windows, and an easy to
use TreeSketch application for iPad. We primarily experimented with using VLAB
under but found that a significant amount of effort was required to install and
interact with the tools and we estimated that there would be a large learning
curve before we would proficient enough to modify the software for our needs.

OpenAlea is implemented as a set of Python libraries utilizing native code for
certain functions. As with Algorithmic Botany, installation and use of the tools
was challenging. OpenAlea appears to be most targeted towards plant researchers
who wish to manually model specific plants.

As L-systems are conceptually very simple, we decided to create a separate
implementation of the algorithms in order to avoid risks associated with reusing
existing software which we have little experience with. Our software was
implemented using Java and OpenGL for drawing. The top-level architecture can be
divided into two main categories: plant simulation using L-systems and the
graphical user interface.

\subsection{Plants}

Our L-systems algorithms are implemented in the LSystem and LRule classes, which
are responsible for storing the production rules associated with each plant and
iterating the plants phenotype as the plant grows throughout the simulation. The
Gene and Plant classes store representations of each living plant. The Gene
class represents the plant genotype as a combination of the L-system production
rules along with additional information such as stem and leaf size, stem and
flower colors, the number of seeds produced, and the age at which the plant
matures. Each individual plants phenotype is represented using the Plant class
and contains it's X/Y position and orientation inside the world, along with the
current age and a L-system string (L-string) representing the plants current
shape. The Plant class is also responsible for calculating each individual
plants fitness values during each simulation tick.

\subsection{User Interface}

The graphical interface is comprised of the Main class, a Drawer class, and a
Video utility class for recording screen captures of the simulation. The Main
class receives user input to control the plant simulation. It also contains the
environmental parameters and invokes the plant growth and fitness functions. The
Drawer class receives a list of plants from the Main class and displays them to
the user. The graphical interfaces also provides convenient controls for each of
the environmental parameters so that they can be tuned while the simulation is
executing.

\onecol{images/architecture.png}{arch}{Conceptual software architecture}

\section{Plant Simulation}

Each plant in our environment contains a string of symbol (which we call an
``L-string'') representing the plants current shape and size. Each symbol can
represent a stem segment, a flower, a leaf, etc. The plants genotype contains a
set of production rules that govern the growth and development of the plant.
When the plant grows, the production rules are used to replace symbols in the
plants current L-string with sequences of tokens to produces a larger and more
complex plant.

\onecol{images/screenshot.png}{screen}{Example plant simulation}

\subsection{L-system Genotypes}

We implemented a modified variant of L-systems where a replacement probability
is included in addition to the matching and replacement parts of each production
rule. The possible values for each symbol in our rules are as follows:

\begin{description}[leftmargin=!,labelindent=0.2in,labelwidth=0.1in]
  \item[G]   grow a stem segment
  \item[F]   grow a flower at the current position
  \item[L]   grow a leaf at the current position
  \item[X]   does not correspond to any drawing action and is used to control
             the evolution of the curve.\cite{lsystems}
  \item[n, e, s, w] \hfill \\
             bend plants growing direction either north, east, south or west.
             The angle is dependent on the genes angle.
  \item[{[}] push matrix, the current position and angle of growth are saved.
  \item[{]}] pop matrix, the position and angle from he corresponding {]} are
             restored.
\end{description}

Using the push and pop symbols, {[} and {]}, the plant can grow branches in
different directions. For example, a north -- south Y shape can be grown using
the string: \[ G [ s G ] [ n G ] \]

Example of a plant at the first generation:

\begin{description}[leftmargin=!,labelindent=0.2in,labelwidth=0.4in]
  \item[Angle] 15 degrees
  \item[Axiom] $ X $
  \item[Rules] $ X \rightarrow Gs[[G]eX]nG[wGX]G \\
                 G \rightarrow GG $
\end{description}

Notice that this plant will not start with flowers ($F$) or leaves ($L$).

\subsection{Simulation}

The simulation of the plant environment is governed by ticks. During each tick,
each of the currently living plant objects becomes older. We wanted to have a
similar approach to how plants grow in the real world. Consequently, during the
first 4 ticks the plant grows as a seed under ground. On the fifth tick it
starts growing above ground. When the seventh tick is reached, the plant becomes
mature and grows flowers and leaves, if the plants genotype is capable of it.
The plant object then spreads it seeds within a certain radius. When the season
is over, the plant dies, but it's children grows up with either the same gene,
or a mutation of the gene.

\section{Genetic Algorithms}

In plant biology, reproduction is divided into two categories: asexual
reproduction and sexual reproduction. In our project we chose to use asexual
reproduction to avoid implementing genetic crossover and recombination. Each of
our plants is initially a clone of it's parent, however there is a probability
that mutations will be introduced during the cloning process.\cite{plantrepo}

In our simulation, the seasons are divided into ticks, at the seventh tick of
the season, the plants are considered mature and will produce seeds. After the
plants have reproduced and the season has ended all the existing plants are
removed and the new child plants are given a chance to grow.

\subsection{Mutation}

When each plant produces seeds, the seed may mutation in several different
ways. Mutation is controlled based on a mutation probability global to all the
plants, if the seed is chosen to be mutated it will change randomly in one of
the following ways:

\begin{description}
  \item[Stem color]   \hfill \\
    The plants stem color is changed randomly, the stem color color is also used
    for the plants leaves.
  \item[Flower color] \hfill \\
    The plants flower color is changed randomly. Both the stem color and flower
    color are only used for display and can be used to help differentiate plants
    of different species.
  \item[Stem width]   \hfill \\
    The diameter of the plants stem is changed. Upper and lower limits are
    imposed and a width is randomly chosen from within the range.
  \item[Leaf size]    \hfill \\
    The length of each leaf is changed. Upper and lower limits are used. The
    leaf shape is fixed for drawing purposes.
  \item[Growth rules] \hfill \\
    The L-system productions used to control the plants growth are changed. The
    plant may become either more or less complex.
\end{description}

\subsubsection{Growth rule mutation}

We want to simulate subtle changes in the gene as the mutation isn't suppose to
be extreme in any ways. Gene mutation occurs by choosing one of the production
rules and then mutating the replacement side of the rule. A symbol is randomly
chosen from the replacement to be mutated.

The mutation can either result in a symbol being deleted or begin replaced with
a sequences from a predefined set of possible mutations.

The replacement sequences we used are logically divided into several logical
groups, however these divisions are just for convenience and are not used by the
mutation algorithm:

\begin{description}[leftmargin=!,labelindent=0.2in,labelwidth=0.5in]
  \item[Stems]   $G,~~xG,~~[G],~~[xG]$
  \item[Flowers] $F,~~xF,~~[F],~~[xF]$
  \item[Leaves]  $L,~~xL,~~[L],~~[xL]$
  \item[Other]   $[X],~~xX$
\end{description}

Note that in in each of the replacements, `x' is substituted with a random
direction to grow, i.e. `n', `e', `s' or `w'.

\subsection{Fitness}

% // Water
% wfit  -= surfaceArea   * water;      // Large surface area looses water
% wfit  += stemRatio     * water;      // Large stems can store water
% 
% // Sun
% sfit  += surfaceArea*2 * sun;        // Large leaves gather more light
% sfit  += stemCount     * sun;        // Taller plants have an advantage
% 
% // Nutrition
% nfit  -= stemRatio     * nutrition;  // Large plants require more nutrients
% nfit  -= surfaceArea   * nutrition;  // Large plants require more nutrients
% nfit  -= stemCount     * nutrition;  // Large plants require more nutrients

At the start of each simulation tick, each existing plant updates a fitness
value based on the current environmental parameters and the plants current
growth state. The list of plants is then sorted based on fitness and unfit
plants are removed if the environment is over crowded.

\subsubsection{Environmental parameters}

There are three basic environmental parameters which have an effect on the
plants.

\begin{description}[leftmargin=!,labelindent=0.2in,labelwidth=0.5in]
  \item[sun]       The amount of sunlight
  \item[water]     The availability of water
  \item[nutrition] The nutritional content of the soil
\end{description}

Each of these parameters initially ranges from 0 to 100 and can be changed by
the user interacting with the software. In addition, a mutation parameter
controls the rate of change in the plant genomes.

For the purpose of fitness calculations, a ``genetic pressure'' is calculated
for of each of these parameters and used in the fitness function.
\[pressure = 1 - (parameter / 100)\]

\subsubsection{Plant parameters}

While evaluating the fitness function, several plant parameters are used and
derived parameters are created to simplify the fitness function itself:

\begin{description}[leftmargin=!,labelindent=0.2in,labelwidth=0.1in]
  \item[Stem ratio]    \hfill \\
    Normalized stem width, 0 = narrow stems, 1 = wide stems
  \item[Leaf ratio]    \hfill \\
    Normalized leaf size, 0 = small leaves, 1 = wide leaves
  \item[Stem count]    \hfill \\
    The number of stem segments the plant has grown thus far.
  \item[Leaf count]    \hfill \\
    The number of leaves grown thus far.
  \item[Storage space] \hfill \\
    The total storage space available to the plant. We do not calculate an
    actual volume but instead use a combination of the stem width from the
    plants genes and the number of stem segments.
    \[ storageSpace = stemRatio \times stemCount \]
  \item[Surface area]  \hfill \\
    The total surface area available in the leaves for photosynthesis. Similar
    to storage space, this is a combination of the leaf size and the number of
    leaves.
    \[ surfaceArea = leafRatio \times leafCount \]
\end{description}

\subsubsection{Fitness calculation}

The plant fitness function is divided into three parts corresponding to the
water, sun, an nutrient levels. Each of these is calculated separately and the
total fitness is summed afterwards. We implemented the fitness equations such
that if all the environmental parameters are equal, then all the fitness values
will also be equal and the plants will evolve randomly. In a more comprehensive
system, some plants could be considered more or less fit in any environment
including an environment with equal environmental parameters.

The following guidelines were used by our fitness function:

\paragraph{Water fitness}
\begin{itemize}
  \item Large stems (high storage space) can conserve water and result in higher
    fitness in dry environments.
  \item A large surface area results in larger water loss and a lower
    fitness in dry environments.
\end{itemize}
\vspace{-0.5em}
\begin{eqnarray*}
waterFitness &=& (storageSpace \times waterPressure) \\
             &-& (surfaceArea \times waterPressure)
\end{eqnarray*}

\paragraph{Sunlight fitness}
\begin{itemize}
  \item A large surface area collects more light and results in a higher fitness
    in shady environments.
  \item More stem segments will result in a taller plant which can college more
    light. We do not calculate actual heights, but assume that additional stem
    segments improve light gathering.
\end{itemize}
\vspace{-0.5em}
\begin{eqnarray*}
sunFitness &=& (2 \times surfaceArea \times sunPressure) \\
           &+& (stemCount \times sunPressure)
\end{eqnarray*}

\paragraph{Nutrition fitness}
\begin{itemize}
  \item Larger stems, more stem segments, and a larger surface all result in
    higher nutrient requirements and thus lower the fitness in low quality soil.
\end{itemize}
\vspace{-0.5em}
\begin{eqnarray*}
nutrientFitness &=&-(surfaceArea \times nutrientPressure) \\
                &+&-(storageSpace \times nutrientPressure) \\
                &+&-(storageSpace \times nutrientPressure)
\end{eqnarray*}

\section{Conclusion}

During our simulations we were able to adjust the environmental parameters in
order to control the growth and evolution of different plant species. In many
cases the evolved plants share some resemblance with natural plants found in
similar environments. Some example of this include stout leafless plants
resembling cacti which evolve in dry environments, large leafy plants resembling
trees that evolve in shady environments, and small shrubby or grass like plants
that evolve in poor soil conditions.

In future programs we would recommend improving the genetic fitness functions by
taking into consideration environmental information related to the neighboring
plants. This would require more advanced lighting and collision detection to
calculate sunlight collection and photosynthesis based on shadows in the
environment.

In addition, more realistic fitness levels could be used that are not linear
with regards to the amount of pressure from different environmental parameters.
In our system, a leafy plant may always win over other plants in a shady
simulation environment even though such as large plant may not be support in a
physical environment to do water or soil limitations.

Some difficulty was encountered evolving plants that appear realistic; some
limitations may be inherit to genetic algorithms which rely on graphical
simulations. For example, two physical plants may share a remarkable resemblance
but may be suited to different environments due to small evolutionary changes.
Waxy leaves, for example, could result in a leafy plant well suited to a dry
environment. Additionally, flowering plants often evolve to attract insects and
other animals which act as pollinators and seed dispersers. Without complex
organisms interacting with the plants it may be difficult to evolve certain
features such as fruits and flowers, without predefining certain
characteristics.

Overall, we have shown that artificial plants based on L-systems can be evolved
in realistic ways based on abstract environmental parameters. Using L-strings to
represent plants results in both realistic looking shapes and provides the
foundation for genetic algorithm for controlling the development of plant
species.

%%%%%%%%%%%%%%
% References %
%%%%%%%%%%%%%%

\bibliographystyle{acmsiggraph}
\bibliography{paper}

\end{document}
